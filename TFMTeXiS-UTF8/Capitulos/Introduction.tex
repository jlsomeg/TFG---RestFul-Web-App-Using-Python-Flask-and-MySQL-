\chapter{Introduction}
\label{cap:introduction}

Recommendation systems are invisible, discrete and opaque elements, that most people don't know (if they exist even) but all people have used sometime in his life. With the rise of the internet and processing capacity, the rise in the amount of data to which a user has access are, unequivocally speaking, infinite. Terms of which it has been spoken for years as information overload (insert reference https://www.reuters.com/article/net-us-technology-mobile-poll/online-sharinginformation- overload-is-worldwide-problem-poll idUSBRE8840NF20120905), or with applications in more recent times such as Big Data, denote the amount of data that users have to deal with when making decisions. \\

Both the problem of saturation, as the solution of the recommendation, are not terms that have been invented in this century. What books Gothic novel should start reading if I want to get into this genre literary, were relatively resolved if you start with the most famous, or better yet, those that can be recommended to you by an acquaintance versed in the genre and who knows in some way your likes and dislikes. The automated recommendation systems devices differ from this known in one thing. The great capacity for processing and obtaining data. Through this, the recommenders are able to recommend elements and guide you through paths that you did not know they existed, but that your interactions denote relationship with them. \\

There are many recommendation systems that are applied in variety of scenarios and environments, most of these related interests with leisure or electronic commerce (e-commerce) since its origins begin ahi (enter reference Brent Smith, Greg Linden, Two Decades of Recommender Systems at Amazon.com, 2017). But in relation to the use of these systems applied to educational systems adapted, little has been developed in a niche yet to be exploited. These scenarios present a problem similar to those seen above. The translation is, in which the amount of learning objects resembles to the information overload that we talked about earlier. \\ 

New technologies and the fast growth of the Internet have made access to information easier for all kind of people, raising new challenges to education when using Internet as a medium. One of the best examples is how to guide students in their learning processes. The need to look for guidance from their teachers or other companions that many Internet users experience when endeavoring to choose their readings, exercises o practises is a very common reality. In order to cater for this need many different information and recommendation strategies have been developed. Recommendation Systems is one of these. Recommendation Systems try to help the user, presenting him those objects he could be more interested in, based on his known preferences or on those of other users with similar characteristics. (enter reference Oscar Sanjuán Martínez, Cristina Pelayo G-Bustelo, Rubén González Crespo, Enrique Torres Franco 2, on the recommendation System for E-learning Environments at degree level, 2009). \\

In this work, we will carry out an investigation that will focus on designing a problem recommendation system for users of the platform online Take On the Challenge that comes closest to the skill level, demand and interests of it and how to implement its functionality in a meta-web application where you can observe the information more remarkable









