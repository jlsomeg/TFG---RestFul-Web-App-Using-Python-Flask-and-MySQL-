\chapter*{Abstract}

¡Acepta el reto! is an online store and judge of programming problems in Spanish that accepts solutions in C, C ++ and Java. Create an academic% environment where learn and practice programming languages ​​and topics and receive feedback on the solutions proposed. But the division of problems by volumes and categories may not be enough to connect with the needs of the user to use this tool.\\

This work arises from the need to involve a system of recommendation of problems to be presented to a user that uses online judge programs, so that in this way,he can choose to solve problems that are included in his level of ability and knowledge or enter more easily into new technical fields, which will have a positive pedagogical impact on his use of the tool and will reduce his level of frustration to the time to raise problems that exceed or underestimate the level of the user. The project differentiates 2 clear parts: \\

	1. Approach, design and development of statistical and mathematical logic to approximate, as much as possible, the skill level and user need to the problems that best suit this at the time \\
	
	2. Design of a high-level online architecture where the developed ideas of the recommender are implemented and where an informative summary can be observed both of the users and of the problems of Acepta el reto! \\


\section*{Keywords}

\noindent Recommender systems, Online Judges, Web Application, Problem resolution,Elo,Rest, Take On the Challenge 



