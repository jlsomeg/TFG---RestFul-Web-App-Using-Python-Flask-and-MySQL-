\chapter*{Resumen}

!Acepta el reto! es un almacén y juez en línea de problemas de programación en español que acepta soluciones en C, C++ y Java. Crea un entorno académico en el que aprender y practicar lenguajes y temáticas de programación y recibir un feedback de las soluciones planteadas. Pero la división de problemas por volúmenes y categorías puede ser no suficiente para conectar con las necesidades del usuario para usar esta herramienta.
 
Este trabajo surge ante la necesidad de involucrar un sistema de recomendación de problemas a plantear ante un usuario que de uso a programas de jueces en linea, para que, de esta manera,   pueda elegir solucionar problemas que estén comprendidos en su nivel de habilidad y conocimiento o adentrase mas fácilmente en nuevos campos técnicos, lo que tendrá un impacto pedagógico positivo en su utilización de la herramienta y disminuirá su nivel de frustración a la hora de plantear problemas que sobrepasan o subestiman el nivel del usuario.El proyecto diferencia 2 partes claras: \\

	1. Planteamiento,diseño y desarrollo de la lógica estadística y matemática para aproximar, lo máximo posible, el nivel de habilidad y necesidad del usuario a los problemas que mas se ajustan a este en el momento \\
	
	2. Diseño de una arquitectura online de alto nivel donde se implementen las ideas desarrolladas del recomendador y donde se pueda observar un resumen informativo tanto de los usuarios como de los problemas de !Acepta el reto! \\
	
-------------------------------aqui me he quedado---------------------------------- 


\section*{Palabras clave}
   
\noindent Máximo 10 palabras clave separadas por comas

   


